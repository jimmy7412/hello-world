\documentclass[twocolumn]{report}
\usepackage[utf8]{inputenc}
\usepackage{graphicx}
\usepackage{amsmath}

\title{Physics 226 Notes}
\author{jimmy7412}

\usepackage{natbib}
\usepackage{graphicx}
\graphicspath{{Images/}}

\begin{document}

\maketitle
\tableofcontents{}

\section*{Introduction}
Notes about Physics 226

\part{First Midterm}

\chapter{Chapter R1 The Principle of Relativity}

\section{The First}

\subsection{R1.1 Introduction to the Principle}

\begin{quote}
    The laws of physics are the same inside a laboratory moving at a constant velocity as they are in laboratory at rest.
\end{quote}

The principle of relativity cannot by proved but it also hasn't been disproved, so it is widely accepted to be true. 

\subsection{R1.2 Events and Spacetime Coordinates}

Event\\
A physical occurrence that can be considered to have happened at a definite instance in spacetime. \footnote{An event is a physical occurrence that can be considered to have happened at a definite instance in space-time}

Space-time coordinates are 4 numbers that tell where and when something is. They are the 3 coordinates X,Y,Z and the time. 

\subsection{R1.3 Reference Frames}

\subsection{R1.4 Inertial Reference Frames}

Inertial Reference Frames

Is a reference frame that is isolated and is always everywhere observed to be moving at a constant velocity.

Non inertial Reference Frame

Is observed to be moving at a non constant velocity for some situations. 

\bibliographystyle{plain}
\bibliography{references}
\end{document}
