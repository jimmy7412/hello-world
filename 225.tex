\documentclass[twocolumn]{article}
\usepackage[utf8]{inputenc}
\usepackage{graphicx}
\usepackage{amsmath}

\title{225 Notes}
\author{James}

\graphicspath{{Images/}}

\begin{document}

\maketitle
\tableofcontents{}
\section{Chapter 1}

$|\psi\rangle=|+\rangle + |-\rangle$ This is a Quantum State $\psi$ it needs the money signs to be in the math mode in order to use it. 

\subsection{Stern-Gerlach Experiments}
Stern-Gerlach Experiments are where a particle (or many particles) is passed through a magnetic field in a certain direction and then they are observed going in a specific direction. 

After a particle goes through an SG it will align itself with that SG's direction. So if a particle is 100\% positive Z then it goes through an X SG, it will then be split in to 50/50 positive and negative Z. 

\begin{figure}[htp]
\centering
\includegraphics[width=6cm]{SternWiki.png}
\caption{An example of a stern Gerlach Experiment}
\label{fig:lion}
\end{figure}

\subsection{Postulate One}
The state of a quantum Mechanical system, including all the information you can know about it, is represented mathematically by a normalized Ket.

\subsection{Quantum State Vectors}

These act like vectors as a basis state for a Hilbert Space.\footnote{Hilbert Space is the space where quantum state vectors live and work.} They are normally in the Z basis or the $S_z$ basis. The general Quantum state vector $\psi$ is the equation that represents a basis state. The $|+\rangle$ and $|-\rangle$ represent orthogonal vectors. The Ket is how they are normally represented and pictured. 

Ket:
\begin{equation}\label{bra}
    |\psi\rangle=a|+\rangle+b|-\rangle
\end{equation}

Conversely there is also a Bra that is like the opposite of a Ket in that it is the complex conjugate of the Ket. All that this means is that the imaginary numbers all go from positive to negative. 

Bra:
\begin{equation}\label{ket}
    \langle\psi|=a^*\langle+|+b^*\langle-|
\end{equation}

Because the vectors $|+\rangle$ and $|-\rangle$ are orthonormal to each other when the inner product of a bra and a ket are taken only the respecting parts are multiplied. 

\begin{equation} \label{eq1}
\begin{split}
\langle+|\psi\rangle & = \langle+|(a|+\rangle+b|-\rangle) \\
 & = \langle+|a|+\rangle+\langle+|b|-\rangle\\
 & = a\langle+|+\rangle+b\langle+|-\rangle\\
 & = a
\end{split}
\end{equation}

When they are flipped they become the complex conjugate of the inner product calculated $\langle+|\psi\rangle$ $\rightarrow$ $\langle\psi|+\rangle^*$. For two Quantum States $\langle\phi|\psi\rangle\rightarrow\langle\psi|\phi\rangle^*$ 

\subsubsection{Example 1.1}

Normalize the vector $|\psi\rangle=C(1|+\rangle+2i|-\rangle)$. The complex constant C is often reffered to as the \textbf{Normalization Constant}.

To normalize $|\psi\rangle$, we set the inner product of the vector with itself equal to unity and then solve for C, note the requisite complex conjugation.

\begin{equation} \label{eq2}
\begin{split}
1 & = \langle\psi|\psi\rangle \\
 & = C^*(1|\langle+|-2i\langle-|)C(1|+\rangle+2i|-\rangle)\\
 & = C^*C(1\langle+|+\rangle+2i\langle+|-\rangle-2i\langle-|+\rangle+4\langle-|-\rangle)\\
 & = |C^2|(1\langle+|+\rangle+4\langle-|-\rangle)\\
 & = 5|C^2|\\
 \frac{1}{5} & = C^2\\
 \sqrt{\frac{1}{5}} & = C
\end{split}
\end{equation}

The overall phase of the normalization constant is not physically meaningful, so we follow the standard convention and choose it to be real and positive. This yields $C=\frac{1}{\sqrt{5}}$. The normalized quantum state is 

\begin{equation}
    |\psi\rangle=\frac{1}{\sqrt{5}}(1|+\rangle+2i|-\rangle)
\end{equation}

\subsection{Postulate 4 (Spin-one half System)}

The probability of obtaining the value $\pm\hbar/2$ in a measurement of the observable $S_z$ on a system in the state $|\psi\rangle$ is

\begin{equation}
    \mathcal{P}_\pm=|\langle\pm|\psi\rangle|^2,
\end{equation}

where $|\pm\rangle$ is the basis ket of $S_z$ corresponding to the result $\pm\hbar/2$

Hello

\end{document}
